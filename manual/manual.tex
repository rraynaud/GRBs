% !TEX TS-program = pdflatex
% !TEX encoding = UTF-8 Unicode

% This is a simple template for a LaTeX document using the "article" class.
% See "book", "report", "letter" for other types of document.

\documentclass[11pt]{article} % use larger type; default would be 10pt

\usepackage[utf8]{inputenc} % set input encoding (not needed with XeLaTeX)
\usepackage[english,francais]{babel}
%%% Examples of Article customizations
% These packages are optional, depending whether you want the features they provide.
% See the LaTeX Companion or other references for full information.

%%% PAGE DIMENSIONS
\usepackage{geometry} % to change the page dimensions
\geometry{a4paper} % or letterpaper (US) or a5paper or....
% \geometry{margin=2in} % for example, change the margins to 2 inches all round
% \geometry{landscape} % set up the page for landscape
%   read geometry.pdf for detailed page layout information

\usepackage{graphicx} % support the \includegraphics command and options

% \usepackage[parfill]{parskip} % Activate to begin paragraphs with an empty line rather than an indent

%%% PACKAGES
\usepackage{booktabs} % for much better looking tables
\usepackage{array} % for better arrays (eg matrices) in maths
\usepackage{paralist} % very flexible & customisable lists (eg. enumerate/itemize, etc.)
\usepackage{verbatim} % adds environment for commenting out blocks of text & for better verbatim
\usepackage{subfig} % make it possible to include more than one captioned figure/table in a single float
\usepackage{amsmath}
\usepackage{natbib}
\usepackage{siunitx}
% These packages are all incorporated in the memoir class to one degree or another...

%%% HEADERS & FOOTERS
\usepackage{fancyhdr} % This should be set AFTER setting up the page geometry
\pagestyle{fancy} % options: empty , plain , fancy
\renewcommand{\headrulewidth}{0pt} % customise the layout...
\lhead{}\chead{}\rhead{}
\lfoot{}\cfoot{\thepage}\rfoot{}

%%% SECTION TITLE APPEARANCE
\usepackage{sectsty}
\allsectionsfont{\sffamily\mdseries\upshape} % (See the fntguide.pdf for font help)
% (This matches ConTeXt defaults)

%%% ToC (table of contents) APPEARANCE
\usepackage[nottoc,notlof,notlot]{tocbibind} % Put the bibliography in the ToC
\usepackage[titles,subfigure]{tocloft} % Alter the style of the Table of Contents
\renewcommand{\cftsecfont}{\rmfamily\mdseries\upshape}
\renewcommand{\cftsecpagefont}{\rmfamily\mdseries\upshape} % No bold!

%%% Abbreviations for typical math environments
\def\be{\begin{equation}}
\def\ee{\end{equation}}
\newcommand{\refig}[1]{Fig.~\ref{#1}}
\newcommand{\reftab}[1]{Table~\ref{#1}}
\newcommand{\tx}[1]{\textrm{#1}}
\newcommand{\Doppler}{\ensuremath{\mathcal{D}}}
\newcommand{\Ndip}{\ensuremath{N_\tx{dip}}}
\newcommand{\Nacc}{\ensuremath{N_\tx{acc}}}
\newcommand{\alfvenradius}{\ensuremath{r_\tx{m}}}
\newcommand{\corotationradius}{\ensuremath{r_\tx{c}}}
\newcommand{\Ldip}{\ensuremath{L_\tx{dip}}}
\newcommand{\Lrad}{\ensuremath{L_\tx{ra}}}
\newcommand{\Lelec}{\ensuremath{L_e}}
\newcommand{\Lbb}{\ensuremath{L_\tx{bb}}}
\newcommand{\Eint}{\ensuremath{E_\tx{int}}}
%%% END Article customizations

%%% The "real" document content comes below...

\title{The G.R.B. code}
\author{J\'er\^ome Guilet, Rapha\"el Raynaud, Matteo Bugli}
%\date{} % Activate to display a given date or no date (if empty),
         % otherwise the current date is printed

\begin{document}
\maketitle

\selectlanguage{francais}
\section{Résumé}
Pour produire des courbes de lumière associées à la coalescence de
deux étoiles à neutrons, le code G.R.B.  utilise un modèle adapté des
travaux de \citet{sun2017} qui suppose que l'objet compact formé est
un magnétar qui perd de l'énergie par un phénomène de freinage
magnétique induit par la composante dipolaire du champ magnétique de
l'étoile. L'évolution temporelle de la fréquence de rotation de
l'étoile est déterminée par le couple magnétique du rayonnement du
dipôle
\begin{equation}
\Omega(t) = \frac{\Omega_0}{\left(1 + t/\tau_{\rm em}\right)^{1/2}},
\end{equation}
où $\tau_{\rm em} = 3c^3 I/(B^2 R^6 \Omega_0^2) $ est le temps
caractéristique de ralentissement, $I$, $R$ et $B$ étant le moment
d'inertie, le rayon et le champ magnétique dipolaire de l'étoile à
neutrons. On suppose que le rayonnement X est produit par dissipation
interne dans le vent du magnétar et émis de façon isotrope
\citep{zhang2013}. La luminosité en rayons X est alors déterminée par
la luminosité de ralentissement du dipôle, modulo un facteur
d'efficacité~$\eta$
\begin{equation}
  L_X = \eta \Ldip(t) = \eta \frac{B^2 R^6 \Omega^4(t)}{6 c^3}
  \,.
\end{equation}
Le magnétar central résultant de la fusion de deux étoiles à neutrons,
il faut également tenir compte de la présence d'éjecta opaques
résiduels qui vont d'abord absorber ce rayonnement dans certaines
directions et le ré-émettre avec un spectre de corps noir. Le calcul
des courbes de lumière dans la zone où le rayonnement est initialement
piégé requiert donc de déterminer l'évolution dynamique des ejecta. On
utilise dans ce cas le modèle de \citet{yu2013}, qui prend en compte
l'injection d'énergie par le magnétar central et le chauffage
additionel par désintégration d'éléments radioactifs.

Enfin, précisons que cette approche phénoménologique permet de
comparer simplement différentes équations d'état qui déterminent le
rayon $R$, le moment d'inertie $I$ et la période critique de rotation
du magnétar en deçà de laquelle l'objet s'effondre en trou noir. Cette
période $P$ est reliée à la masse maximale d'une étoile à neutrons par
la relation \citep{lasky2014}
\begin{equation}
  M_\tx{max} = M_\tx{TOV} (1+ \alpha P^\beta)
  \,,
\end{equation}
où la masse maximale d'une étoile statique $M_\text{TOV}$ et les
exposants $\alpha$ et $\beta$ sont donnés par les modèles d'équation
d'état \citep{ai2018}. Lorsque l'étoile s'effondre en trou noir du
fait du ralentissement de sa rotation, on suppose que la luminosité
due au vent du magnétar s'arrête abruptement.



\selectlanguage{english}
\section{Equations}

We consider a magnetar of radius $R$, mass $M$, moment of inertia $I$
and surface magnetic field $B$, surrounded by an accretion disc with
initial mass~$M_\tx{disc}$ and radius~$R_\tx{disc}$. The time
evolution of the magnetar angular frequency $\Omega$ will be
determined both by the magnetic and accretion torques \Ndip{} and
\Nacc{}, defined respectively by
\begin{align}
  \Ndip &= - \frac{\mu^2 \Omega^3}{6 c^3} \,,\\
  \Nacc &= n(\Omega) \left(G M \alfvenradius \right)^{1/2} \dot{M}
  \,,
\end{align}
where $\mu = B R^3$ is the dipole moment and \alfvenradius{} the
Alfvén radius
\begin{equation}
  \alfvenradius = \mu^{4/7} \left(GM\right)^{-1/7} \dot{M}^{-2/7}
  \,.
\end{equation}
Depending of the values of the Alfvén radius and the corotation
radius~$\corotationradius = \left(GM/\Omega^2\right)^{1/3}$, the
system will be accreting or expelling material: if $\alfvenradius <
\corotationradius$, the system is accreting and the accretion torque
spins up the magnetar, whereas for $\alfvenradius > \corotationradius$
the magnetar loses angular momentum with the expelled material --- the
so-called propeller regime \citep{gompertz2014}. The change of sign of
the accretion torque is handle by the following prefactor
\begin{equation}
  n(\Omega) =
  \begin{cases}
    \begin{aligned}
      &1 - \left(\frac{\alfvenradius}{\corotationradius}\right)^{3/2}  &\alfvenradius> R\\
      &1 - \frac{\Omega}{\Omega_\tx{K}}  &\alfvenradius < R
    \end{aligned}
  \end{cases}
  \,,
\end{equation}
where $\Omega_\tx{K}=\sqrt{GM/R^3}$.

\begin{itemize}
\item BAR MODE INSTA
\item accretion rate
\item ejecta
\end{itemize}


The complete ODE system to be integrated in time is then
\citep{sun2017}
\begin{align}
 \dot{\Omega} &= \frac{\Ndip + \Nacc}{I} \,,\\
 \dot{t}'     &= \Doppler(\Gamma) \,,\\
 \dot{\Gamma} &= \frac{(\Ldip + \Lrad - \Lelec)-
   \tfrac{\Gamma}{\Doppler}(\xi \Ldip + \Lrad - \Lelec)+
   \Gamma\Doppler\tfrac{\Eint'}{3V'}(4\pi R^2\beta c)}
     {M_\tx{ej}c^2+ \Eint'} \,,\\
 \dot{E}_\tx{int}' &= \Doppler\left[\tfrac{1}{\Doppler^2}(\xi \Ldip + \Lrad - \Lelec)-
   \tfrac{\Eint'}{3V'}(4\pi R^2\beta c)\right] \,,\\
\dot{V}' &= 4\pi R^2\beta c \Doppler \,,\\
\dot{R} &= \frac{\beta c}{1-\beta}
\,.
\end{align}
In the above system, \Doppler{} is the Doppler factor defined by
$\Doppler = [\Gamma(1-\beta\cos\theta)]^{-1}$ with $ \beta(\Gamma) =
(1-\Gamma^{-2})^{-1/2}$ and $\xi$ an efficiency parameter defining the
fraction of the spin-down energy that is used to heat the ejecta. The
different luminosities that enter the equations are the dipole
spin-down luminosity~\Ldip{}, the co-moving radiative heating
luminosity~\Lrad{} and the co-moving bolometric emission luminosity of
the heated electrons~\Lelec{}, respectively
\begin{align}
  \Ldip &= \frac{B^2 R^6 \Omega(t)^4}{6c^3} \,,\\
  \Lrad &=  4 \times 10^{49} M_\tx{ej,-2}
  \left[\frac{1}{2}-\frac{1}{\pi} \arctan
    \left(\frac{t'-\SI{1.3}{s}}{\SI{0.11}{s}}\right)
    \right]^{1.3} \Doppler^2 \,,\\
  \Lelec' &=
  \begin{cases}
    \begin{aligned}
    & c\Eint' \tfrac{\Gamma}{\tau R} &t < t_{\tau=1} \\[1ex]
    & c\Eint' \tfrac{\Gamma}{R} &t \geq t_{\tau=1}
    \end{aligned}
  \end{cases}
  \,.
\end{align}
Optical depth of the ejecta
\begin{equation}
  \tau = \kappa \frac{M_\tx{ej}}{V'}\frac{R}{\Gamma}
\end{equation}
Blackbody spectrum
\begin{align}
T' &=
\begin{cases}
  \begin{aligned}
    &\left(\frac{\Eint'}{a V' \tau}\right)^{1/4} & \tau > 1 \\
    &\left(\frac{\Eint'}{a V'}\right)^{1/4} & \tau \leq 1
  \end{aligned}
\end{cases} \,,\\
\nu \Lbb &= \frac{8 \pi^2 \Doppler^2 R^2}{h^3 c^2}
\frac{(h \nu/\Doppler)^4}{\exp\left(h\nu/\Doppler kT'\right)-1}
\,.
\end{align}
where $a$ and $k$ are respectively the blackbody radiation constant
and the Boltzmann constant\footnote{In CGS units, we have
  $a=\SI{7.5646e-15}{erg.cm^{-3}.K^{-4}}$ and
  $k=\SI{1.380658e-16}{erg/K}$.}.


The main output of the code are the X-ray luminosities for both the
free and trapped zones
\begin{align}
  L_\tx{X,free}(t) &= \eta \Ldip(t) \,,\\
  L_\tx{X,trapped}(t) &= e^{-\tau} \eta \Ldip(t)
  + \int_{\SI{0.3}{keV}}^{\SI{6}{keV}} \nu \Lbb d \nu
  \,.
\end{align}
where one needs to introduce the factor $\eta$ to parametrize the
efficiency at which the dipole spin-down luminosity is converted to
X-ray luminosity.

\begin{itemize}
\item magnetar collapse
\end{itemize}

The code has been benchmark against the results of
\citet{gompertz2014}.


\bibliographystyle{bib/aa}
\bibliography{bib/grb}

\end{document}
